\usepackage{ifthen}
\usepackage[normalem]{ulem} % for \sout
\usepackage{xcolor}
\usepackage{amssymb}

\newcommand{\ra}{$\rightarrow$}
\newboolean{showedits}
\setboolean{showedits}{true} % toggle to show or hide edits
\ifthenelse{\boolean{showedits}}
{
	\newcommand{\ugh}[1]{\textcolor{red}{\uwave{#1}}} % please rephrase
	\newcommand{\ins}[1]{\textcolor{blue}{\uline{#1}}} % please insert
	\newcommand{\del}[1]{\textcolor{red}{\sout{#1}}} % please delete
	\newcommand{\chg}[2]{\textcolor{red}{\sout{#1}}{\ra}\textcolor{blue}{\uline{#2}}} % please change
}{
	\newcommand{\ugh}[1]{#1} % please rephrase
	\newcommand{\ins}[1]{#1} % please insert
	\newcommand{\del}[1]{} % please delete
	\newcommand{\chg}[2]{#2}
}

\newboolean{showcomments}
\setboolean{showcomments}{true}
%\setboolean{showcomments}{false}
\newcommand{\id}[1]{$-$Id: scgPaper.tex 32478 2010-04-29 09:11:32Z oscar $-$}
\newcommand{\yellowbox}[1]{\fcolorbox{gray}{yellow}{\bfseries\sffamily\scriptsize#1}}
\newcommand{\triangles}[1]{{\sf\small$\blacktriangleright$\textit{#1}$\blacktriangleleft$}}
\ifthenelse{\boolean{showcomments}}
%{\newcommand{\nb}[2]{{\yellowbox{#1}\triangles{#2}}}
{\newcommand{\nbc}[3]{
 {\colorbox{#3}{\bfseries\sffamily\scriptsize\textcolor{white}{#1}}}
 {\textcolor{#3}{\sf\small$\blacktriangleright$\textit{#2}$\blacktriangleleft$}}}
 \newcommand{\version}{\emph{\scriptsize\id}}}
{\newcommand{\nbc}[3]{}
 \renewcommand{\ugh}[1]{#1} % please rephrase
 \renewcommand{\ins}[1]{#1} % please insert
 \renewcommand{\del}[1]{} % please delete
 \renewcommand{\chg}[2]{#2} % please change
 \newcommand{\version}{}}
\newcommand{\nb}[2]{\nbc{#1}{#2}{orange}}

\definecolor{ibcolor}{rgb}{0.4,0.6,0.2}
\newcommand\iv[1]{\nbc{IB}{#1}{ibcolor}}
\usepackage{wasysym}
\newcommand\yesml[1]{\nbc{ML {\textcolor{yellow}\sun}}{#1}{mircolor}}

%% here you can make your own annotations
\definecolor{sgcolor}{rgb}{0.2,0.0,0.5}
\newcommand\sg[1]{\nbc{SG}{#1}{sgcolor}}

\definecolor{hccolor}{rgb}{0.21,0.54,0.84}
\newcommand\hc[1]{\nbc{HC}{#1}{hccolor}}

\definecolor{ideacolor}{rgb}{1.0,0,0.5}
\newcommand\idea[1]{\nbc{IDEA}{#1}{ideacolor}}


\definecolor{abstractcolor}{rgb}{0.0,0.5,1.0}
\newcommand\rabstract[1]{\nbc{ABSTRACT}{#1}{abstractcolor}}

\definecolor{introcolor}{rgb}{0.0,1.0,0.5}
\newcommand\rintro[1]{\nbc{INTRO}{#1}{introcolor}}

\definecolor{papercolor}{rgb}{1.0,1.0,0.0}
\newcommand\rpaper[1]{\nbc{PAPER}{#1}{papercolor}}

\definecolor{multicolor}{rgb}{1.0,0,0}
\newcommand\rmulti[1]{\nbc{MULTI}{#1}{multicolor}}

% Todo Command
\definecolor{todocolor}{rgb}{0.9,0.1,0.1}
\newcommand{\todo}[1]{\nbc{TODO}{#1}{todocolor}}

