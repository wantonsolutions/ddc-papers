\section{Motivations}


The case for resource disaggregation is relatively straightforward. Scalability,
energy savings, density and dynamic composition of systems are the primary
benefits. Given a disaggregated model it’s possible to compose a system which
can meet arbitrary user requirements. Perhaps the easiest to understand is that
a server can be provisioned with more ram than will fit into a single box. From
a data center operator perspective one of the biggest advantages is that you
don't need to actually upgrade a machine in order to change the provisioning
that are being offered. Section 2.1 of \textit{Can Far Memory Improve Job
Throughput} explains this point in detail~\cite{jobthroughput}. 

Another bleeding edge but relevant advantage towards disaggregated designs is
that they take into account current trends hardware accelerators. Traditional
designs would have accelerators either be co-located in a sever provisioned
with memory and compute, with the resources accessed via some RPC API. In the
disaggregation model accelerators can be placed in racks specially designed for
them. An example of this being an advantage is that a naive machine learning
practitioner does not need to learn how to use something like TensorFlow to make
use of a cluster. The GPU's can all be accessed as if local via usual single
machine libraries. This assumes virtual memory mapping. The high level take away
is that as computation becomes more heterogeneous across accelerators
disaggregated designs provide a paradigm that allows them to be interfaced with
at mass scale using traditional OS primitives.

Many people have argued that disaggregation could provide significant energy
savings. At a high level, resources can be powered down when they are not being
used. I'm not sure how true this is, but Zombie land at least makes the
case~\cite{zombieland}.

\section{Problems}

Resource disaggregation essentially redesigns how a computer is composed, and
therefore it is fraught with new problems to solve. Mainly in terms of
performance, failures, and density.

\textbf{Performance}

Memory is the biggest performance concern, and the one which requires the most
innovation. The standard measure of performance degradation is 20x. Memory is
roughly 50ns and remote memory is 1us. This are numbers actually tilted in the
favor of remote memory, few projects have gone sub 1us and it's hard to call
them disaggregated. Given that the cost of going remote is high there are a few
areas where the research could help improve performance.

Transparent far memory systems, i.e using remote memory for paging can definitely
improve their prefecting algorithms. The vast majority of prefecting analysis
has been done to optimize for spinning disks, and do things like attempt to
prefetch large contiguous chunks of memory. Leap~\cite{leap} demonstrates that
just by looking at stride patterns they can get a serious performance advantage.
Projects like ASC~\cite{asc} have demonstrated that address prediction can be
done with high accuracy, techniques that use additional, potentially offline
computation to predict good prefetching might be justified when the cost is so
high. Prefeched pages also do network transit and can generate extreme traffic
demands. Congestion control and priority will be required here. A quick example
is that the page which a processor is waiting for to execute the next
instruction should be given high priority, while predicted prefetched pages
should be given low priority. Lastly the size of the accessed memory and how it
is accessed is an open question. VM ware has suggested that remote memory be
accessed at the cache line granularity not at the page granularity. The same
concerns apply, but also concerns around small packets.



\textbf{Failures}

Resource disaggregation introduces new classes of failures. The most common
example is how to deal with a portion of an address space disappearing.
Solutions include adding redundancy for memory such as duplicating, or using
erasure coding. Another model suggests exposing failures as exceptions to
applications which they can handle, similar but not equal to a page fault. Other
suggest different computational models such as explicit checkpointing. 

No such models have been built or simulated to my knowledge. There are a variety
of trade offs to any approach here and many projects could be pitched around how
to structure guarantees, API, and programming modes for cloud native
disaggregated computing. The most promising proposals are arguably coming from
the serverless community which suggest making all computation compartmentalized.
Disaggregation and the application~\cite{ddcaps} has the best argument for
traditional signaling.

\textbf{Density}

There are serious questions surrounding how to interact with new degrees of
density. For instance, where does the kernel live, and how is it interacted
with? Normally there is one kernel per box, but now there might be one kernel
per rack. This could save a lot of memory overall. It raises interesting OS
design questions about what a context switch would look like i.e memory
protection might be more easily implemented in network and hardware rings of
protection might not be nessisary. This also requires a redesign of how
scheduling and interrupts would be handled.

Questions around loss are also relevant. For instance, the scale of losses given
proposed disaggregation 


\textbf{Data Structures} Data structures for remote memory are a key requirement
for disaggregated systems. The motivation here is that normal data structures
tank in performance not due to their algorithm complexity but their poor
consideration of the system. For instance a linked list in remote memory suffers
from pointer chasing penalties. Every access to each element requires a round
trip to resolve an address, this destroys performance. It shows it's head most
clearly when resources are contested.

Similar to NUMA optimized data structures there is going to be an obvious
benifit to far memory data structures by partitioning data and designing them to
be associative and commutative whenever possible. Remote memory is different than
NUMA in one main aspect -- Data transits the network, and the network has small
amounts of compute. Far memory data structures can take advantage of in network
compute for serious gains. See my work for reducing contention using a switch.

The considerations here are similar to designing cache aware data structures,
but inverted because the network inverts the memory hierarchy. i.e between L3 and
main there is a unified L2 (NIC/FPGA memory) and L1 switch memory. It's a memory
diamond rather than pyramid, rather than designing for cache locality the data
structures need to be designed for operation aggregation on small amounts of
data. For instance any operations that can be performed by a switch, i.e
updating the next write address are good, because they avoid the endhosts from
communicating and provide full throughput. Put concisely data structures with
structural invariant which are assertable using a fraction of the data
structures size are desirable. There is very little research here.

